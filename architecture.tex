\documentclass[11pt]{article}
\addtolength{\oddsidemargin}{-1.cm}
\addtolength{\textwidth}{2cm}
\addtolength{\topmargin}{-2cm}
\addtolength{\textheight}{3.5cm}

\usepackage[pdftex]{graphicx}
\usepackage{pdflscape}
\usepackage{hyperref}
\usepackage{float}
\usepackage{cite}
\hypersetup{
	colorlinks=true,
	linkcolor=black,
	filecolor=magenta,
	urlcolor=cyan,
}

% define the title
\author{Team Python}
\title{Architectural Design}

\begin{document}
	\setlength{\parskip}{6pt}
	\begin{titlepage}
	\newcommand{\HRule}{\rule{\linewidth}{0.5mm}} % Defines a new command for the horizontal lines, change thickness here

	\center % Center everything on the page
	 
	%----------------------------------------------------------------------------------------
	%	LOGO SECTIONS
	%----------------------------------------------------------------------------------------

	\includegraphics[width=\textwidth]{front-page}

	%----------------------------------------------------------------------------------------
	%	TITLE SECTION
	%----------------------------------------------------------------------------------------

	\HRule \\[0.4cm]
	{ \huge \bfseries Software Requirements Specification}\\[0.4cm] % Title of your document
	\HRule \\[1.5cm]
	 
	%----------------------------------------------------------------------------------------
	%	MEMBERS, TEAM NAME SECTION
	%----------------------------------------------------------------------------------------

	\begin{minipage}{0.5\textwidth}
	\begin{flushleft} \large
	\emph{Members:}\\% add your name and student here
	Dewald de Jager  		\hfill	15188800 \\
	Johan du Plooy    		\hfill	12070794 \\
	Juan du Preez 			\hfill	15189016 \\
	Marthinus Hermann		\hfill	15081479 \\
	Orisha Orrie 			\hfill	13025199 \\
	Azhar Patel 				\hfill	15052592 \\
	Maria Qumayo   			\hfill	29461775 \\
	\end{flushleft}
	\end{minipage}
	~
	\begin{minipage}{0.4\textwidth}
	\begin{flushright} \large
	{ \huge \bfseries Team Python }% Title of document
	{\large \today}\\
	{\large v0.1}
	\end{flushright}
	\end{minipage}\\[4cm]
\end{titlepage}

	\tableofcontents
	\newpage
	
	\section{Introduction}
    \subsection{Purpose}
	\subsection{Scope}
	
	\section{External Interface Requirements}
	
	\section{Performance Requirements}
	\paragraph{The system should be able to run in the following way: }
	\begin{itemize}
	\item The system will be run through campus servers as well as web servers.
	\item The performance of the application will be dependent on the user’s hardware device.
	\item The loading time of each page will be dependent on the strength of the user’s internet connection as well as the number of users currently using campus Wi-Fi.
	\item The time it takes to log in will be dependent on the internet connection.
	\item All new users should be added to the system in less than a second however, it may take the user longer to see this, depending on the internet connection.
	\item Users should be sent notifications as soon as they are available on the system.
	\item Admin should be able to access the GIS module as well as user module with ease in less than a second as these details will be located on campus servers.
	\item Points of interest should appear as soon as a user searches for a location or searches for a specific point of interest however, this will once again be dependent on the internet connection.
	\item Users should be able to search, delete, save as well as modify their route. This should be able to take place quickly. Saving and modifying will be dependent on the user and searching and deleting will be dependent on the internet connection.
	\item Rendering display service will differ from device to device while requesting service relies on the internet connection.
	\item Push services for events should happen as soon as the event has been added to the application
	\item Fitness service and heat maps should be updated in real time.
	\end{itemize}
	
	\section{Design Constraints}
	
	\section{Software System Attributes}
	\subsection{Reliability}
	\paragraph{When transferring data from point A to point B, it is crucial that the integrity of every piece of information be maintained and accounted for.  One piece of misinformation could very well mean the difference between the locations of two different class rooms or even completely different buildings.  One of the applications features is to ensure the fastest route, which relies heavily on making the right decisions based on the data it receives, therefore if the information is not accurate or even correct, the user could end up late to their desired destination. }
	\subsection{Robustness}
	\subsection{Efficiency}
	\paragraph{Sending data between remote locations relies heavily on the network traffic and its capabilities/limitations.  Thus the idea would be to put the least amount of stress and congestion on that network and only make strategic requests from client to server.  The idea would be to make relatively frequent requests to small yet crucial information, and much less frequent requests to information that is less likely to change within a small period of time.}
	\subsection{Interoperability}
	\subsection{Maintainability}
	\subsection{Testability}
	\subsection{Portability}
	\subsubsection{Hardware Independence}
	\paragraph{Seeing as how the system will be deployed with a web base as a back-end API, the hardware choices are not limited to a specific set of components, so long as they meet the minimum or preferably recommended requirements to be able to handle the estimated network traffic to keep the system performing at optimal levels.}
	\subsubsection{Software Dependence}
	\paragraph{The software on the other hand will be limited to being web specific, i.e. being able to use or integrate with applications like AJAX, Node etc.}
	\subsection{Reusability}
	\subsection{Modularity}
	\subsection{Cohesion}
	\subsection{Coupling}
	The subsystems should have minimal impact on and dependency on the other subsystems of the system. The system as a whole applies the low coupling principle. Subsystems should be abstracted and implementation changes should only affect the subsystem that was modified. This abstraction layer will provide an interface between all the subsystems that prevents interference and augments communication between them.
	
	The data module will be represented as a module with requests and responses that are used both by access channels and the server. This allows for other subsystems to easily use it without having to worry about implementation specific details and the implementation can be changed to use a different technology or update the application programming interface (API).
	
	\section{Technologies}
	\subsection{Data}	
	\subsection{Users}
	\subsection{Events}
	\subsection{Points of Interest}
	
	\section{UML Diagrams}
	\subsection{Data}
	\subsection{Users}
	\subsection{Events}
	\subsection{Points of Interest}
\end{document}
